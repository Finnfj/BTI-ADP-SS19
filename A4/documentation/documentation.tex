\documentclass[11pt]{scrartcl}

\usepackage{ucs}
\usepackage[utf8x]{inputenc}
\usepackage{ngerman}
\usepackage{amsmath,amssymb,amstext}
\usepackage{graphicx}
\usepackage[automark]{scrpage2}
\usepackage{pgfplots}
\usepackage{chngcntr}
\usepackage[left=2cm, right=2cm, top=2cm]{geometry}
\counterwithin{figure}{section}

\pagestyle{scrheadings}

\title{Pascalsches Dreieck}
\author{Finn Jannsen, Philipp Schwarz}
\date{\today{}}

\begin{document}

\maketitle

\tableofcontents

\section{Einführung}
	\label{sec:einfuehrung}
	
	Diese Dokumentation beschreibt drei Implementations-Varianten zur Berechnung einer Reihe aus dem Pascalschen Dreieck.
	Hierbei wurde eine Rekursive, eine Iterative und eine Strategie, die mittels Binominalkoeffizient berechnet angewandt.
	In Abschnitt \ref{sec:implementation} wird darauf eingegangen, wie die verschiedenen Varianten realisiert wurden.
	Anschließend wird in Abschnitt \ref{sec:verfun} geprüft, ob die Varianten korrekt funktionieren und in Abschnitt \ref{sec:aufwand} die Performance der Varianten verglichen.

\section{Implementation}
	\label{sec:implementation}
	
	Die drei Varianten wurden in Java als Klassen implementiert. Da alle den gleichen Funktionssatz brauchen implementieren sie alle das gleiche Interface namens PascalTriangleCalculator.\\
	Da die Klassenstrukturen im wesentlichen nur eine Methode beinhalten, die die gewünschte Reihe berechnet und als Integer-Array zurückgibt, 
	liegt im Folgenden der Fokus auf der Implementierung der Algorithmen.


	\subsection{Rekursive Strategie}
		\label{sec:recStrat}
		
		Rekursive Strategie beschreiben
	
	\subsection{Iterative Strategie}
		\label{sec:iterStrat}
		
		Iterative Strategie beschreiben
	
	\subsection{Binominalkoeffizient Strategie}
		\label{sec:binoStrat}
		
		Binominalkoeffizient Strategie beschreiben

\section{Verifizieren und Testen}
\label{sec:vertests}

	\subsection{Verifizieren der Funktionalität}
		\label{sec:verfun}
		
		Alle Strategien wurden auf ihre Funktionalität durch einen Vergleich von ein paar resultierenden Reihen und von Hand errechneten Ergebnissen überprüft.
		Alle Tests wurden erfolgreich absolviert.
	
	\subsection{Aufwandsanalyse}
		\label{sec:aufwand}
		
		Aufwand Analysieren blabla

		Rechenoperationen bei iterativer und rekursiver Strategie
		\begin{equation*}
		R = \frac{n^2}{2}+\frac{n}{2}
		\end{equation*}

		Rechenoperationen bei Binominalkoeffizient-Strategie
		\begin{equation*}
		R = n^2+4*n
		\end{equation*}

\begin{figure}
	\begin{center}
		\newcommand{\binoCol}{brown}
		\newcommand{\recCol}{blue}
		\newcommand{\iterCol}{red}
		\newcommand{\posMark}{square*}
		\newcommand{\keyMark}{*}
		\begin{tikzpicture}
			\begin{loglogaxis}[
					title={\large Berechnung Operationen},
					height=10cm,
					width=17cm,
					grid=major,
					x tick label style={
					/pgf/number format/1000 sep=},
					ylabel=Rechenoperationen,
					xlabel=Reihe,
					enlargelimits=0.05,
					legend style={at={(0.5,-0.15)},
					anchor=north,legend columns=-1},
				]
				\addplot[color=\binoCol,mark=\keyMark]
				    coordinates {(1,1) (2,2) (4,4) (8, 32) (16, 184) (32, 879) (64, 3805) (128, 15835) (256, 64475) (512, 260059) (1024, 1044443) (2048, 4186075) (4096, 16760795) (8192, 67076059)
					(16384, 268369883)};
				\addplot[color=\recCol,mark=\keyMark]
				    coordinates {(1,1) (2,3) (4,10) (8, 36) (16, 136) (32, 528) (64, 2080) (128, 8256) (256, 32896) (512, 131328) (1024, 524800) (2048, 2098176) (4096, 8390656) (8192, 33558528)
					(16384, 134225920)};
				\addplot[color=\iterCol,mark=\keyMark]
				    coordinates {(1,1) (2,3) (4,10) (8, 36) (16, 136) (32, 528) (64, 2080) (128, 8256) (256, 32896) (512, 131328) (1024, 524800) (2048, 2098176) (4096, 8390656) (8192, 33558528)
					(16384, 134225920)};
				\legend{Binominalkoeffizient,Rekursiv, Iterativ}
			\end{loglogaxis}
		\end{tikzpicture}
		\caption{Quantitativer Vergleich der Berechnung einer Reihe im Pascalschen Dreieck in loglog-Darstellung}
	\end{center}
\end{figure}

\begin{figure}
	\begin{center}
		\newcommand{\binoCol}{brown}
		\newcommand{\recCol}{blue}
		\newcommand{\iterCol}{red}
		\newcommand{\posMark}{square*}
		\newcommand{\keyMark}{*}
		\begin{tikzpicture}
			\begin{axis}[
					title={\large Berechnung Zeit},
					height=10cm,
					width=17cm,
					grid=major,
					ylabel=Zeit in ms,
					xlabel=Reihe,
					enlargelimits=0.05,
					legend style={at={(0.5,-0.15)},
					anchor=north,legend columns=-1},
				]
				\addplot[color=\binoCol,mark=\keyMark]
				    coordinates {(1,0) (2,0) (4,0) (8, 0) (16, 0) (32, 1) (64, 0) (128, 1) (256, 2) (512, 3) (1024, 1) (2048, 7) (4096, 26) (8192, 105)
					(16384, 418) (32768,1676) };
				\addplot[color=\recCol,mark=\keyMark]
				    coordinates {(1,0) (2,0) (4,0) (8, 0) (16, 0) (32, 0) (64, 0) (128, 0) (256, 0) (512, 2) (1024, 8) (2048, 10) (4096, 38) (8192, 102)
					(16384, 419) (32768,1688) };
				\addplot[color=\iterCol,mark=\keyMark]
				    coordinates {(1,0) (2,0) (4,0) (8, 0) (16, 0) (32, 0) (64, 0) (128, 1) (256, 2) (512, 2) (1024, 6) (2048, 11) (4096, 41) (8192, 129)
					(16384, 454) (32768,1818) };
				\legend{Binominalkoeffizient,Rekursiv, Iterativ}
			\end{axis}
		\end{tikzpicture}
		\caption{Quantitativer Zeitvergleich der Berechnung einer Reihe im Pascalschen Dreieck}
	\end{center}
\end{figure}

\end{document}


%%% Local Variables:
%%% mode: latex
%%% TeX-master: t
%%% End:

%
%\begin{figure}[h!]
%\texttt{ \\
%// Index 0 is reserved for stopper, but previousIndex always pointing to last Pos-Container in row\\
%int i=0;    \\
%while(true) {\\
%    counter++;\\
%    int pre = positions[i].getPreviousIndex();\\
%    Elem tmp = (Elem)positions[pre].getPointer();\\
%    if (tmp != null) {\\
%        if (tmp.key == key) {\\
%            return positions[pre];\\
%        }\\
%    } else {\\
%        //error\\
%        return null;\\
%    }\\
%    i = pre;\\
%}\\
%}
%\caption{Ausschnitt Such-Mechanik Container-Array}
%\label{figure:findmech}
%\end{figure}