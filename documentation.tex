\documentclass[11pt]{scrartcl}

\usepackage{ucs}
\usepackage[utf8x]{inputenc}
\usepackage{ngerman}
\usepackage{amsmath,amssymb,amstext}
\usepackage{graphicx}
\usepackage[automark]{scrpage2}
\usepackage{pgfplots}
\usepackage{chngcntr}
\counterwithin{figure}{section}

\pagestyle{scrheadings}

\title{Implementation des ADT Set}
\author{Finn Jannsen, Philipp Schwarz}
\date{\today{}}

\begin{document}

\maketitle

\tableofcontents

\section{Einführung}
\label{sec:einfuehrung}

Diese Dokumentation beschreibt drei Implementations-Varianten des Abstrakten Datentyps Set. Dieser Datentyp soll, sofern möglich, eine Menge darstellen.
Als Vorgabe zur Art der Implementation wurde 1. ein Array von Elementen, 2. ein Array von Containern und 3. eine verkettete Liste von Containern angeführt.
In Abschnitt \ref{sec:implementation} wird darauf eingegangen, wie die verschiedenen Varianten realisiert wurden.
Anschließend prüfen wir in Abschnitt \ref{sec:verifikation}, ob die vorgegebenen Operationen funktionieren und vergleichen in Abschnitt \ref{sec:aufwand} die Performance der Varianten.

\section{Implementation}
\label{sec:implementation}

\subsection{Set als Array}
\label{sec:setArray-I}

Implementation beschreiben

\subsection{Set als Array von Containern}
\label{sec:setConArray-I}

Implementation beschreiben

\subsection{Set als einfach verkettete Liste von Containern}
\label{sec:setCon-I}

Implementation beschreiben

\section{Verifizieren und Testen}
\label{sec:vertests}

\subsection{Verifizieren der Funktionalität}
\label{sec:verfun}
Da die drei Varianten sich nur intern im Aufbau unterscheiden und dies nach außen hin gekapselt ist, verifizieren wir ihre Funktion anhand einer Menge an gleichen Tests.
Die in Tabelle ~\ref{table:vertests} enthaltenen Tests soll eine ausreichende Verifizierung der Operationen auf den Sets ermöglichen.

\begin{table}
\begin{tabular}[ht]{|l|p{12cm}|}
\hline
T1             & \(add : SET \times ELEMENT \to SET \times POS\) \\ \hline
Beschreibung  & Hinzufügen eines Elements \(e \in ELEM\) in die Menge \(s \in SET\)           \\ \hline
pre-condition  & -           \\ \hline
post-condition & \texttt{s.find(e.key).isValid = TRUE}           \\ \hline
\hline
T2             & \(delete : SET \times POS \to SET\) \\ \hline
Beschreibung  & Entfernen eines Elements \(e \in ELEM\) in Position \(p \in POS\) aus der Menge \(s \in SET\)           \\ \hline
pre-condition  & \texttt{p.getSet() == s \&\& s.retrieve(p) == e \&\& p.isValid = TRUE}            \\ \hline
post-condition & \texttt{p.isValid = FALSE}           \\ \hline
\hline
T3             & \(delete : SET \times KEY \to SET\) \\ \hline
Beschreibung  & Entfernen eines Elements \(e \in ELEM\) in Position \(p \in POS\) mit Schlüssel \(k \in KEY\) aus der Menge \(s \in SET\)           \\ \hline
pre-condition  & \texttt{s.find(k).getSet() == s \&\& s.retrieve(s.find(k)) == e \&\& s.find(k).isValid = TRUE}            \\ \hline
post-condition & \texttt{p.isValid = FALSE}           \\ \hline
\hline
T4             & \(find : SET \times KEY \to POS\) \\ \hline
Beschreibung  & Suchen der Position \(p \in POS\) eines Elements \(e \in ELEM\) mit Schlüssel \(k \in KEY\) aus der Menge \(s \in SET\)           \\ \hline
pre-condition  & -           \\ \hline
post-condition & Wenn \(k \in s\): \texttt{s.find(k).isValid = TRUE}. Wenn \(k \notin s\): \texttt{s.find(k).isValid = FALSE}           \\ \hline
\hline
T5             & \(retrieve : SET \times POS \to ELEM\) \\ \hline
Beschreibung  & Zugriff auf Element \(e \in ELEM\) in Position \(p \in POS\) aus der Menge \(s \in SET\)           \\ \hline
pre-condition  & \texttt{p.getSet() == s \&\& p.isValid = TRUE}           \\ \hline
post-condition & \texttt{s.retrieve(p) == e}           \\ \hline
\hline
T6             & \(size : SET \to INTEGER\) \\ \hline
Beschreibung  & Mächtigkeit der Menge \(s \in SET\)           \\ \hline
pre-condition  & -           \\ \hline
post-condition & \texttt{s.size() == |s|}           \\ \hline
\hline
T7             & \(unify : SET \times SET \to SET\) \\ \hline
Beschreibung  & Vereinigung zweier Mengen \(s_{1},s_{2} \in SET\)           \\ \hline
pre-condition  & -           \\ \hline
post-condition & \texttt{s.unify(s1, s2) == }\(s_{1} \cup s_{2}\)           \\ \hline
\end{tabular}
\caption{Verifikationstests}
\label{table:vertests}
\end{table}

Die in der Tabelle ~\ref{table:vertests} aufgeführten Tests wurden mit verschiedenen Parametern und Quantitäten in verschiedenen Sequenzen in JUnit4 getestet. 
Alle drei Implementations-Varianten erfüllen die oben genannten Tests mit pre- und post-condition.

\subsection{Aufwandsanalyse}
\label{sec:aufwand}

\begin{figure}
\newcommand{\bestCol}{brown}
\newcommand{\avgCol}{blue}
\newcommand{\worstCol}{red}
\newcommand{\posMark}{square*}
\newcommand{\keyMark}{*}
\begin{tikzpicture}
\begin{loglogaxis}[
	title={\large Container Test-Diagramm},
	height=15cm,
	width=15cm,
	grid=major,
    x tick label style={
        /pgf/number format/1000 sep=},
    ylabel=Operationen,
    xlabel=Listengröße,
    enlargelimits=0.05,
    legend style={at={(0.5,-0.1)},
    anchor=north,legend columns=-1},
]
\addplot[color=\bestCol,mark=\keyMark]
    coordinates {(100000,4) (10000,4)
         (1000,4) (100,4) (10, 4)};
\addplot[color=\avgCol,mark=\keyMark]
    coordinates {(100000,150004) (10000,15004)
         (1000,1504) (100,154) (10, 19)};
\addplot[color=\worstCol,mark=\keyMark]
    coordinates {(100000,300001) (10000,30001)
         (1000,3001) (100,301) (10, 31)};
\addplot[color=\bestCol,mark=\posMark]
    coordinates {(100000,2) (10000,2)
         (1000,2) (100,2) (10, 2)};
\addplot[color=\avgCol,mark=\posMark]
    coordinates {(100000,100002) (10000,10002)
         (1000,1002) (100,102) (10, 12)};
\addplot[color=\worstCol,mark=\posMark]
    coordinates {(100000,200000) (10000,20000)
         (1000,2000) (100,200) (10, 20)};
\legend{KeyBest,KeyAverage, KeyWorst, 
PosBest,PosAverage, PosWorst}
\end{loglogaxis}
\end{tikzpicture}
\caption{delete(Pos) und delete(Key) in Container-Liste}
\end{figure}

\end{document}


%%% Local Variables:
%%% mode: latex
%%% TeX-master: t
%%% End:
